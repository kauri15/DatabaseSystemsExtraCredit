\documentclass{article}
\usepackage{graphicx} 
\usepackage{amsmath}
\usepackage{amssymb}

\title{Database Systems HW 1 for Question 4}
\author{Ishneet Kaur}
\date{February 3 2026}

\begin{document}

\maketitle
\section{Question 4}
You are given the following set $F$ of functional dependencies for a relation $R(A, B, C, D, E, F)$.
\[
F = \{ A \rightarrow D,\; C \rightarrow BD,\; BD \rightarrow E,\; E \rightarrow CDF \}
\]

Is $AC \rightarrow CEF$ implied by $F$? In other words, is $AC \rightarrow CEF \in F^+$?
If yes, use the inference rules to show how you can obtain this functional dependency from the
dependencies in $F$. 
\\\\

\subsection*{Solution}

Yes, $AC \rightarrow CEF$ is implied by $F$. This can be shown by computing the closure of $AC$ and by using Armstrong’s inference rules.

\subsubsection*{Part 1: Compute $(AC)^+$}
With $A \rightarrow D$, we can add $D$:
\[
(AC)^+ = \{A, C, D\}
\]

With C now being in the closure, we can use the functional dependency $C \rightarrow BD$ to add $B$:
\[
(AC)^+ = \{A, B, C, D\}
\]

With both attributes B and D in the closure, we can use the functional dependency $BD \rightarrow E$ to add $E$:
\[
(AC)^+ = \{A, B, C, D, E\}
\]

With this updated closure, we can now use the functional dependency $E \rightarrow CDF$ to add $F$:
\[
(AC)^+ = \{A, B, C, D, E, F\}
\]

Since all attributes are included in $(AC)^+$ including CEF, we conclude:
\[
AC \rightarrow CEF \in F^+
\]

\subsubsection*{Part 2: Inference Rules}
We will now examine if the inference rules confirm $AC \rightarrow $ CEF \\\\
We will start with a given functional dependency. \\\\
Given: $A \rightarrow $ D \\\\
Next, we can use augmentation to add C to both sides as we want the left side to represent AC. \\\\
Augmentation: $AC \rightarrow $ CD\\\\
We can now decompose this to consolidate to the variables we are looking for on the right side. \\\\
Splitting: $AC \rightarrow $ C\\\\
With C now on the right side alone, we can use another given functional dependency. \\\\
Given: $C \rightarrow $ BD\\\\
We can now apply transitivity. \\\\
Transitivity: $AC \rightarrow $ C and $C \rightarrow $ BD so $AC \rightarrow $ BD \\\\
This now gives us another given functional dependency on the right. \\\\
Given: $BD \rightarrow $ E\\\\
We can apply transitivity again. \\\\
Transitivity: $AC \rightarrow $ BD and $BD \rightarrow $ E so $AC \rightarrow $ E \\\\
Now with all of the implications we have seen for AC, we can combine them to the right side. \\\\
Combining: $AC \rightarrow $ C and $AC \rightarrow $ BD and $AC \rightarrow $ E so $AC \rightarrow $ BCDE \\\\
To get the last attribute, F, we can use another given functional dependency. For this, we can split to get the E which is on the left side of the functional dependency that will give us F. \\\\
Splitting: $AC \rightarrow $ E\\\\
Given: $E \rightarrow $ CDF\\\\
Transitivity: $AC \rightarrow $ E and $E \rightarrow $ CDF so $AC \rightarrow $ CDF \\\\
We can now combine everything together and see everything that AC implies. \\\\
Combining: $AC \rightarrow $ BCDE and $AC \rightarrow $ CDF so $AC \rightarrow $ BCDEF \\\\
To confirm the final answer, we can use splitting to get the exact attributes we are trying to validate. \\\\
Splitting: $AC \rightarrow $ CEF\\\\

\subsubsection*{Conclusion}

Therefore, $AC \rightarrow CEF$ is implied by the given set of functional dependencies so 
\[
AC \rightarrow CEF \in F^+
\]

\end{document}

